\section{Introduction}
Efficient numerical modelling of elastic wave propagation in complex 3D anisotropic domains is a significant challenge in exploration seismology. Large-scale inversion methods such as Full-waveform inversion (FWI) experience difficulties to account for both elasticity and anisotropy \citep{kamath2017elastic}, since the inversion process increases in algorithmic complexity and computational expense. As FWI requires terabytes of data to image both anisotropy and elasticity within Earth's subsurface accurately, the majority of computational cost originates from numerical solutions of elastic wave equations and their corresponding adjoints over large domains \citep{witte2019large}. Traditionally, numerical solutions are created through explicit finite-difference (FD) stencil codes \citep{virieux1986p} typically written in low-level languages such as Fortran or C before manual performance optimisation for specific computer architecture. However, this in itself is an extremely challenging task since FD stencils are usually handwritten and therefore highly error-prone, hence a significant amount of time is spent for both code verification and optimisation. The resulting productivity bottleneck spans both industry as well as academia \citep{lange2016devito}. Multiple domain scientists spend months or even years \citep{louboutin2018devito} performing this cumbersome workflow of creating, verifying and optimising innovative higher-order FD stencils even before evaluation on HPC hardware.\\
\\
This project intends to automate this workflow by using Devito to decouple the problem from its low-level implementation \citep{kukreja2016devito}. Devito is a domain-specific language (DSL) capable of generating highly-optimised low-level stencil code from high-level symbolic equations composed in Python \citep{luporini2018architecture}. The project aims to create a practical forward modelling framework by leveraging Devito's capabilities to implement, verify and benchmark an efficient high-accuracy staggered-grid (SG) \citep{bansal2008finite, tarrass2011new} FD scheme \citep{xu2019modeling} for solving the elastic wave equations in 3D vertical transversely isotropic (VTI) media.
\subsection{Background}
A computational framework for discretising elastic wave equations in 3D anisotropic media is an import requirement in seismic inversion problems, considering most sedimentary rocks within the crust exhibit seismic anisotropy \citep{wang2018numerical}. Generally, FD stencil algorithms are developed to numerically discretise partial differential equations (PDEs) and create solvers for both forward and adjoint wave equations. The stencil is defined by the grid locations and the coefficients of the FD scheme. A forward solver can simulate synthetic data for a specified source and receiver geometry and underlies most FWI frameworks \citep{FWI_part1}. Here, the stencil algorithm computes the value of the forward wavefield field in one spatial location according to the neighbouring ones \citep{louboutin2017performance}. However, most FWI implementations are limited to acoustic approximations \citep{kamath2016elastic}. This is because optimised higher-order acoustic FD stencils are possible to implement over a few months manually. Anisotropic acoustic algorithms, however, do not correctly handle reflection amplitudes and cannot be applied to multi-component data \citep{kamath2016elastic}. Hence, higher-order FD schemes that account for both elasticity and anisotropy are needed. Elastic-VTI FD schemes can account for these parameters and accurately model reflection coefficients \citep{kamath2017elastic} while being computationally cheaper than Elastic TTI solvers \citep{bube2016self}.
Nevertheless, developing a time-stepping algorithm for anisotropic wavefield, solution of the elastic wave-equations requires sophisticated operators \citep{yan2009elastic} resulting in elaborate implementation requirements. The FD stencil here is algorithmically more complex requiring manual computation of FD coefficients and low-level code optimisation. This task necessitates expertise ranging from geophysics, numerical methods and High-Performance Computing (HPC) and typically takes teams of domain scientists and HPC specialists years to complete.  \\
\\
Today, the use of high-level abstractions and symbolic reasoning through DSLs is a widely used approach to reduce the time taken to implement and verify individual operators for wave propagation \citep{louboutin2018devito}. In the context of forward modelling, Devito provides a high-level approach to FD and can automate the process of building forward solver. Through symbolic computation, Devito can dynamically generate the required optimised stencil computation kernels \citep{lange2016devito} to solve the 3D elastic VTI wave equation. i.e. Users express the PDE symbolically, using Sympy, and Devito generates optimised parallel C++ code that precisely defines the mathematics \citep{lange2017optimised}. The Devito Compiler is responsible for lowering process the high-level equations down to parallelised C++ Code \citep{luporini2018architecture}.
Additionally, to speed up Python computation, the Devito Compiler employs Just-In-Time (JIT) compilation to focus on array-oriented computation while utilising LLVM to compile remaining Python code at runtime \citep{lange2016devito}. Furthermore, the compiler also automates the task of performance optimising the generated code. The generated code undergoes three types of performance optimisations; parallelism, data locality and FLOP reduction \citep{luporini2018architecture, louboutin2020scaling}. The user has the flexibility to configure the compiler to use a variety of platform-specific backends, such as YASK (Yet-another-stencil-kernel) to generate optimised C++ code for Intel Xeon and Intel Xeon Phi architectures \citep{luporini2018architecture}, in order to realise further performance. Through this high-level approach to FD, the Devito DSL is genuinely disruptive as it removes the need to focus on the tedious low-level implementation of the 3D elastic VTI FD stencil, thus creating a separation of concerns between domain scientists and HPC specialists to enable a direct payoff in productivity \citep{lange2016devito}.



